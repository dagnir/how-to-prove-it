\documentclass{article}
\usepackage{amsmath}
\usepackage{amsfonts}
\usepackage[top=1in, bottom=1.25in, left=1.25in, right=1.25in]{geometry}
\usepackage{tikz}

\title{Chapter 2}
\author{Dongie Agnir}
\date{April 29, 2016}

\begin{document}
\maketitle

\section*{2.1 Exercises}
\begin{enumerate}
\item
	\begin{enumerate}
	\item Let $F(x,y)$ stand for ``$x$ has forgiven $y$'' and $S(x)$ stand for ``$x$ is a saint.''
		\begin{equation*}
			\forall x (\exists y F(x,y) \Rightarrow S(x))
			\end{equation*}

	\item Let $C(x)$ stand for ``$x$ is in calculus class'', $D(x)$ stand for ``$x$ is in discrete math'', and $S(x,y)$ stand for ``$x$ is smarter than $y$''.
		\begin{equation*}
			\lnot \exists x (C(x) \land \forall y (D(y) \Rightarrow S(x,y)))
		\end{equation*}
	\item Let $L(x,y)$ stand for ``$x$ likes $y$'' and $m = \text{Mary}$.

		\begin{equation*}
			\forall x (x \neq m \Rightarrow L(x,m))
		\end{equation*}
	\item Let $P(x)$ stand for ``$x$ is a police officer'', $S(x,y)$ stand for ``$x$ saw $y$'', $j = \text{Jane}$, and $r = \text{Roger}$.

		\begin{equation*}
			\exists x (P(x) \land S(j,x)) \land \exists y (P(y) \land S(r,y))
		\end{equation*}
	\item Using the same definitions as above:

		\begin{equation*}
			\exists x (P(x) \land S(j,x) \land S(r,x))
		\end{equation*}
	\end{enumerate}
\item
	\begin{enumerate}
	\item Let $B(x)$ stand for ``$x$ bought a Rolls Royce with cash'', $R(x)$ stand for ``$x$ is rich'', and $U(x,y)$ stand for ``$x$ is the uncle of $y$.

		\begin{equation*}
			\forall x (B(x) \Rightarrow \exists y(U(y,x) \land R(y)))
		\end{equation*}
  \item Let $D(x)$ stand for ``$x$ lives in the dorm'', $F(x,y)$ stand for ``$x$ is friends with $y$'', $M(x)$ stand for ``$x$ has measles'' and $Q(x)$ stand for ``$x$ needs to be quarantined''.
    \begin{equation*}
      \exists x(D(x) \land M(x)) \Rightarrow \forall y(\exists z (D(z) \land F(y,z)) \Rightarrow Q(y))
    \end{equation*}

  % 2.1.2 c
  \item Let $F(x)$ stand for ``$x$ failed the test'', $A(x)$ stand for ``$x$ got an A'', $D(x)$ stand for ``$x$ got a D'', and $T(x,y)$ stand for ``$x$ has to tutor $y$.

    \begin{equation*}
      \lnot \exists x F(x) \Rightarrow \forall x (A(x) \Rightarrow \exists y (D(y) \land T(x,y)))
    \end{equation*}
  % 2.1.2 d
  \item Let $D(x)$ stand for ``$x$ can do it'' and $j = \text{Jones}$.
    \begin{equation*}
      \forall x D(x) \Rightarrow D(j)
    \end{equation*}

  % 2.1.2 e
  \item
    \begin{equation*}
      D(j) \Rightarrow \forall x D(x)
    \end{equation*}
	\end{enumerate}
  % 2.1.3
\item
  \begin{enumerate}
  \item
    $\forall n (n > x \Rightarrow n > y)$.  The free variables here are $x$ and $y$.
  \item
    $\forall a (ax^2 + 4x - 2 = 0 \iff a \geq -2)$.  The free variable is $x$.
  \item
    $\forall x (x^3 - 3x < 3 \Rightarrow x < 10)$.  There are no free variables.
  \item
    $(\exists x (x^2 + 5x = w) \land \exists y (4 - y^2 = w)) \Rightarrow -10 < w < 10$.  The free variable is $w$.
  \end{enumerate}
% 2.1.4
\item
  \begin{enumerate}
    \item ``All men who are not married are unhappy.''
    \item ``Someone is an aunt.''
  \end{enumerate}
% 2.1.5
\item
  \begin{enumerate}
  \item ``All prime numbers other than 2 are odd.''
  \item ``There is a perfect number that is larger than all other perfect numbers.''
  \end{enumerate}
% 2.1.6
\item
  \begin{enumerate}
  \item This means there is one person $x$ who is the parent of everyone.  This is false.
  \item This means that everyone has a parent.  This is true.
  \item This means that there is someone who has no parent.  This is false.
  \item This means that there is someone who has no child.  This is true.
  \item This means there are two people that don't have a parent child relationship.  This is true.
  \end{enumerate}
% 2.1.7
\item
  \begin{enumerate}
  \item This says that for all values of $x$, there is a corresponding value $y$ such that $2x - y = 0$; or in other words that double a natural number is also a natural number. This is true.
  \item This says that there is some value $y$ so that for all values of $x$ $2x - y = 0$. This is false.
  \item This says that all natural numbers are even. This is false.
  \item This says that for all values of $x$ that are less than 10, if $y$ is less than $x$ then it's less than $9$.  This is true.
  \item This says there are two numbers that add up to 100. This is true.
  \item This says that for some any number $x$, and a number $y$ that is greater than it, there is a value $z$ such that $y + z = 100$. The only way this would be true is if we could have negative numbers but we don't (our universe is $\mathbb{N}$) so this is false.
  \end{enumerate}
% 2.1.8
\item
  \begin{enumerate}
  \item Double a real number is also a real number.  This is true.
  \item This says there is a real number $y$ such that for any real number $x$, $2x = y$.  This is false.
  \item This says that any real number $x$ can be written as the product of 2 and another real number $y$.  This is true, since we can have fractional numbers.
  \item This says if a real number $x$ is less than 10, then all real numbers $y$ that are less than $x$ are less than 9.  This is false since we can take $x = 9.99$, and $y = 9.98$.
  \item This says that there are two real numbers that add to 100.  This is true.
  \item For any real number $x$, there is a real number $y$ strictly larger than $x$, and another real number $z$ that add to 100.  This is true.
  \end{enumerate}
% 2.1.9
\item
  \begin{enumerate}
  \item This says double an integer is also an integer.  This is true.
  \item This says there is an integer $y$ such that for any nteger $x$, $2x = y$.  This is false.
  \item This says that any integer can be written as the product of another integer and 2.  This is false.
  \item This says that if an integer $x$ is less than 10, then all integers $y$ that are less than $x$ are less than 9.  This is true.
  \item This says there are two integers that add to 100.  This is true.
  \item This says that for any integer $x$, there is an integer $y$ strictly larger than $x$ and another real number $z$ that add to 100.  For $\mathbb{N}$, this was false because we needed negatives for this to be true, so this is true for $\mathbb{Z}$.
  \end{enumerate}
\end{enumerate}
\end{document}
