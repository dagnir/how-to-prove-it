\documentclass{article}
\usepackage{amsmath}
\usepackage[top=1in, bottom=1.25in, left=1.25in, right=1.25in]{geometry}

\title{Chapter 1}
\author{Dongie Agnir}
\date{December 8, 2015}

\begin{document}
\maketitle

\section*{1.1 Exercises}
\begin{enumerate}
\item
  \begin{enumerate}
  \item If we let $P$ stand for the statement ``We'll have a reading assignment'' and $Q$ stand for ``We'll have homework problems'' then we can represent the statement as $(P \lor Q) \land \lnot (P \land Q)$.
  \item If we let $P$ stand for the statement ``You will go skiing'' and $Q$ stand for ``There will be snow'', then we can represent the statement as $\lnot P \lor (P \land \lnot Q)$.
    \item $\lnot [(\sqrt{7} < 2) \lor (\sqrt{7} = 2)]$
  \end{enumerate}
\item
  \begin{enumerate}
  \item Let $P$ stand for the statement ``John is telling the truth'' and $Q$ stand for the statement ``Bill is telling the truth'', then we can represent the statement as $(P \land Q) \lor (\lnot P \land \lnot Q)$.
  \item Let the letters $P$, $Q$, and $R$ stand for the statements ``I will have chicken'', ``I will have fish'' and ``I will have mashed potatoes'' respectively, then the statment can be represented as $(P \lor Q) \land \lnot (Q \land R)$.
  \item Let the $P$, $Q$, and $R$ stand for the statements ``$3$ divides $6$'', ``$3$ divides $9$'', and ``$3$ divides $15$'' respectively, then the statement can be represented as $P \land Q \land R$.
  \end{enumerate}
\item For the following, we let $P$ stand for ``Alice is in the room'' and $Q$ stand for ``Bob is in the room''.
  \begin{enumerate}
  \item $\lnot P \lor \lnot Q$
  \item $\lnot P \land \lnot Q$
  \item $\lnot P \lor \lnot Q$
    \item $\lnot P \land \lnot Q$
  \end{enumerate}
\item Only (a) and (c) are well-formed.	 (b) and (c) both contain statements that are not connected by a logical symbol.
\item
  \begin{enumerate}
  \item I will not both buy the pants and not buy the shirt.
  \item I will not buy the pants or shirt.
  \item Either I will not buy the pants, or not buy the shirt.
  \end{enumerate}
\item
  \begin{enumerate}
  \item Either Steve is happy or George is happy, but not both.
  \item Either Steve is happy or George is happy and not Steve, or George is not happy.
  \item Either Steve is happy or George is happy and either Steve or George is not happy.
  \end{enumerate}
\item
  \begin{enumerate}
  \item The \textbf{premise} are: ``Jane and Pete won't both win the math prize'', Pete will win either the math prize or the chemistry prize'', and Jane will win the math prize.''

    The \textbf{conclusion} is ``Therefore, Pete will win the math prize.''

    Let $P$ stand for ``Pete will win the math prize'', $Q$ stand for ``Pete will win the chemistry prize'', and $R$ stand for ``Jane will win the math prize''.

    We can represent the premises in the order given as $\lnot (P \land R)$, $P \lor Q$, and $R$.

    We can represent the conclusion as $Q$.

    Given the premises, yes the conclusion seems reasonable.

  \item The \textbf{premises} are: ``The main course will be either beef or fish.'', ``The vegetable will be either peas or corn.'', and ``We will not have both fish as a main course and corn as a vegetable''.

    The \textbf{conclusion} is ``Therefore we will not have both beef as a main course and peas a vegetable.''

    Let $A$ stand for ``The main course will be beef'', $B$ stand for ``The main course will be fish'', $C$ stand for ``The vegetable will be peas'', and $D$ stand for ``The vegetable will be corn''.

    We can represent the premises as $(A \lor B)$, $(C \lor D)$, and $\lnot (B \land D)$.

    We can represent the conclusion as $\lnot (A \land C)$.

    No, this conclusion does not seem reasonable.  Given the first two premises, there are four possible combinations of main dish and vegetable.  The third premise rules out one of them, but this only concerns fish with peas.

  \item The \textbf{premises} are: ``Either John or Bill is telling the truth.'', ``Either Sam or Bill is lying.''.

    The \textbf{conclusion} is: ``Therefore, either John is telling the truth or Sam is lying.''

    Let $P$ stand for ``John is telling the truth.'', $Q$ stand for ``Bill is telling the truth.'', and $R$ stand for ``Sam is telling the truth.''.

    We can represent the premises as $P \lor Q$ and $\lnot R \lor \lnot Q$.

    We can represent the conclusion as $P \lor \lnot R$.

    Yes, this seems reasonable to me.

  \item
    The \textbf{premise} is ``Either sales will go up and the boss will be happy, or expenses will go up and the boss will be unhappy.''

    The \textbf{conclusion} is ``Sales and expenses will not go up.''

    Let $P$ stand for ``Sales will go up'', $Q$ stand for ``The boss will be happy'', and $R$ stand for ``Expenses will go up''.

    We can represent the premise as $(P \land Q) \lor (R \land \lnot Q)$.

    We can represent the conclusion as $\lnot (P \lor R)$.

    This doesn't seem reasonable because it's possible for $P$ and $Q$ to be true and $R$ to be false, leaving the premise true, and the conclusion false.
    \end{enumerate}
\end{enumerate}

\section*{1.2 Exercises}
\begin{enumerate}
\item
  \begin{enumerate}
  \item
    \begin{tabular}{c c c c}
      $P$ & $Q$ & $\lnot P$ & $\lnot P \lor Q$ \\ \hline
      T & T & F & T \\
      T & F & F & F \\
      F & T & T & T \\
      F & F & T & T \\
    \end{tabular}
  \item
    \begin{tabular}{c c c c c c c}
      $S$ & $G$ & $S \lor G$ & $\lnot S$ & $\lnot G$ & $\lnot S \lor \lnot G$ & $(S \lor G) \land (\lnot S \lor \lnot G)$ \\ \hline
      T & T & T & F & F & F & F \\
      T & F & T & F & T & T & T \\
      F & T & T & T & F & T & T \\
      F & F & F & T & T & T & F \\
    \end{tabular}
  \end{enumerate}
\item
  \begin{enumerate}
  \item
    \begin{tabular}{c c c c c c}
      $P$ & $Q$ & $\lnot P$ & $Q \lor \lnot P$ & $P \land (Q \lor \lnot P)$ & $\lnot [P \land (Q \lor \lnot P)]$ \\ \hline
      T & T & F & T & T & F \\
      T & F & F & F & F & T \\
      F & T & T & T & F & T \\
      F & F & T & T & F & T \\
    \end{tabular}
\item
    \begin{tabular}{c c c c c c c}
      $P$ & $Q$ & $R$ & $P \lor Q$ & $\lnot P$ & $\lnot P \lor R$ & $(P \lor Q) \land (\lnot P \lor R)$ \\ \hline
      T & T & T & T & F & T & T \\
      T & T & F & T & F & F & F \\
      T & F & T & T & F & T & T \\
      T & F & F & T & F & F & F \\
      F & T & T & T & T & T & T \\
      F & T & F & T & T & T & T \\
      F & F & T & F & T & T & F \\
      F & F & F & F & T & T & F \\
      \end{tabular}
  \end{enumerate}
\item
  \begin{enumerate}
  \item
    \begin{tabular}{c c c}
      $P$ & $Q$ & $P + Q$ \\ \hline
      T & T & F \\
      T & F & T \\
      F & T & T \\
      F & F & F \\
    \end{tabular}
  \item $P + Q$ is equivalent to $(\lnot P \land Q) \lor (P \land \lnot Q)$.  This makes intuitive sense because $+$ is true precisely when only either $P$ or $Q$ is true.  Here is the truth table:

    \begin{tabular}{c c c c c c c}
      $P$ & $Q$ & $\lnot P$ & $\lnot Q$ & $\lnot P \land Q$ & $P \land \lnot Q$ & $(\lnot P \land Q) \lor (P \land \lnot Q)$ \\ \hline
      T & T & F & F & F & F & F \\
      T & F & F & T & F & T & T \\
      F & T & T & F & T & F & T \\
      F & F & T & T & F & F & F \\
      \end{tabular}
  \end{enumerate}
\item
  $P \lor Q$ is equivalent to $\lnot (\lnot P \land \lnot Q)$.	Again, this makes intuitive sense because $\lor$ is only false when both $P$ and $Q$ are false.	 We can also think about slightly differently: DeMorgan's law says $\lnot (P \land Q) \equiv \lnot P \lor \lnot Q$, so if we start off by negating both $P$ and $Q$ first we have $\lnot (\lnot P \land \lnot Q) \equiv \lnot\lnot P \lor \lnot\lnot Q \equiv P \lor Q$.  Here is the truth table:

  \begin{tabular}{c c c c c c}
    $P$ & $Q$ & $\lnot P$ & $\lnot Q$ & $\lnot P \land \lnot Q$ & $\lnot (\lnot P \land \lnot Q)$ \\ \hline
    T & T & F & F & F & T \\
    T & F & F & T & F & T \\
    F & T & T & F & F & T \\
    F & F & T & T & T & F \\
  \end{tabular}
\item
  \begin{enumerate}
  \item
    \begin{tabular}{c c c}
      $P$ & $Q$ & $P \downarrow Q$ \\ \hline
      T & T & F \\
      T & F & F \\
      F & T & F \\
      F & F & T \\
      \end{tabular}
      \item
	This truth table shows clearly that it's just the opposite of the one for \textit{or}, so it's equivalent to $\lnot (P \lor Q) \equiv \lnot P \land \lnot Q$.  This makes sense because \textit{nor} should only be true when both $P$ and $Q$ are false.
      \item
	$\lnot P \equiv P \downarrow P$.  This is very useful because we know see that a statement \textit{nor} itself is equivalent to its negation.

	We know that \textit{nor} is the negation of \textit{or}, so if we negate a \textit{nor} then we get a \textit{or}.  Therefore $P \lor Q \equiv \lnot (P \downarrow Q) \equiv (P \downarrow Q) \downarrow (P \downarrow Q)$.  The third equivalence comes directly from our above result.

	$P \downarrow Q \equiv \lnot P \land \lnot Q$, so if we negate $P$ and $Q$ first, we get $\lnot P \downarrow \lnot Q \equiv \lnot \lnot \land P \lnot \lnot Q \equiv P \land Q$.  So, using ony \textit{nor}: $\lnot P \downarrow \lnot Q \equiv (P \downarrow P) \downarrow (Q \downarrow Q) \equiv P \land Q$.
  \end{enumerate}
\item
  \begin{enumerate}
  \item
    \begin{tabular}{c c c}
    $P$ & $Q$ & $(P \uparrow Q)$ \\ \hline
    T & T & F \\
    T & F & T \\
    F & T & T \\
    F & F & T \\
    \end{tabular}
  \item
    $P \uparrow Q \equiv \lnot P \land \lnot Q$.
  \item
    $\lnot P \equiv P \uparrow P$.

    We know $P \uparrow Q \equiv \lnot P \land \lnot Q$, so $\lnot (P \uparrow Q) \equiv \lnot (\lnot P \land \lnot Q) \equiv P \lor Q \equiv (P \uparrow Q) \uparrow (P \uparrow Q)$.

    $P \uparrow Q \equiv \lnot P \land \lnot Q$, so negating $P$ and $Q$ first: $\lnot P \uparrow \lnot Q \equiv \lnot \lnot P \land \lnot \lnot Q \equiv P \land Q \equiv (P \uparrow P) \uparrow (Q \uparrow Q)$.

    Interestingly, these look identical to the results from problem 5.
  \end{enumerate}
  \item
  \begin{enumerate}
  \item
    For this, we determined that the premises are $\lnot (P \land R)$, $P \lor Q$, and $R$, and the conclusion is $Q$.	Here is the truth table:

    \begin{tabular}{c c c c c c}
      $P$ & $Q$ & $R$ & $P \land R$ & $\lnot (P \land R)$ & $P \lor Q$ \\ \hline
      T & T & T & T & F & T \\
      T & T & F & F & T & T \\
      T & F & T & T & F & T \\
      T & F & F & F & T & T \\
      F & T & T & F & T & T \\
      F & T & F & F & T & T \\
      F & F & T & F & T & F \\
      F & F & F & F & T & F \\
    \end{tabular}

    As we can see, all three premises are true on row 5, and so is the conclusion, so this argument is valid.

  \item
    The premises are $(A \lor B)$, $(C \lor D)$, and $\lnot (B \land D)$, and the conclusion is $\lnot (A \land C)$.

    Here is the truth table.  To save time and space, I only have the premises and conclusion.

    \begin{tabular}{c c c c}
      $A \lor B$ & $C \lor D$ & $\lnot (B \land D)$ & $\lnot (A \land C)$ \\ \hline
      T & T & F & F \\
      T & T & T & F \\
      T & T & F & T \\
      T & F & T & T \\
      T & T & T & F \\
      T & T & T & F \\
      T & T & T & T \\
      T & F & T & T \\
      T & T & F & T \\
      T & T & T & T \\
      T & T & F & T \\
      T & F & T & T \\
      F & T & T & T \\
      F & T & T & T \\
      F & T & T & T \\
      F & F & T & T \\
    \end{tabular}

    As we can see, all three premises are true for rows 2, 5, 6, 7, and 10, but of those rows, the conclusion is true only for row 7 and 10.  This argument is invalid.

  \item The premises are $P \lor Q$ and $\lnot R \lor \lnot Q$, and the conclusion is $P \lor \lnot R$.

    Here is the truth table for the premises and conclusion:

    \begin{tabular}{c c c}
      $P \lor Q$ & $\lnot R \lor \lnot Q$ & $P \lor \lnot R$ \\ \hline
      T & F & T \\
      T & T & T \\
      T & T & T \\
      T & T & T \\
      T & F & F \\
      T & T & T \\
      F & T & F \\
      F & T & T \\
    \end{tabular}

    The premises are both true for rows 2, 3, 4, and 6, and for those rows the conclusion is also true so this argument is valid.
  \item
    The premise is $(P \land Q) \lor (R \land \lnot Q)$, and the conclusion is $\lnot P \land \lnot R$.

    Here is the truth table:

    \begin{tabular}{l r}
      $(P \land Q) \lor (R \land \lnot Q)$ & $\lnot (P \land R)$ \\ \hline
      T & F \\
      T & T \\
      T & F \\
      F & T \\
      F & T \\
      F & T \\
      T & T \\
      F & T \\
    \end{tabular}

    The premise is true for rows 1, 2, 3, and 7, but of those rows, the conclusion is true only for rows 2 and 7, so this argument is invalid.
  \end{enumerate}
\item
  The truth table:

  \begin{tabular}{c c c c c c c}
    $P$ & $Q$ & $(P \land Q) \lor (\lnot P \land \lnot Q)$ & $\lnot P \lor Q$ & $(P \lor \lnot Q) \land (Q \lor \lnot P)$ & $\lnot(P \lor Q)$ & $(Q \land P) \lor \lnot P$ \\ \hline
    T & T & T & T & T & F & T \\
    T & F & F & F & F & F & F \\
    F & T & F & T & F & F & T \\
    F & F & T & T & T & T & T \\
  \end{tabular}

  Looking at the table, we can see that $(P \land Q) \lor (\lnot P \land \lnot Q) \equiv (P \lor \lnot Q) \land (Q \lor \lnot P)$, and $\lnot P \lor Q \equiv (Q \land P) \lor \lnot P$.
\item
  The truth tables:

  \begin{tabular}{c c c c c}
    $P$ & $Q$ & $(P \lor Q) \land (\lnot P \lor \lnot Q)$ & $(P \lor Q) \land (\lnot P \land \lnot Q)$ & $(P \lor Q) \lor (\lnot P \lor \lnot Q)$ \\ \hline
    T & T & F & F & T \\
    T & F & T & F & T \\
    F & T & T & F & T \\
    F & F & F & F & T \\
  \end{tabular}

  \begin {tabular}{c c c c}
    $P$ & $Q$ & $R$ & $[P \land (Q \lor \lnot R)] \lor (\lnot P \lor R)$ \\ \hline
    T & T & T & T \\
    T & T & F & T \\
    T & F & T & T \\
    T & F & F & T \\
    F & T & T & T \\
    F & T & F & T \\
    F & F & T & T \\
    F & F & F & T \\
  \end{tabular}

  We can see that (a) is neither, (b) is a contradiction, and (c) and (d) are tautologies.

\item
  \begin{enumerate}
  \item

    \begin{tabular}{c c c c c c c}
      $P$ & $Q$ & $(P \lor Q)$ & $\lnot (P \lor Q)$ & $\lnot P$ & $\lnot Q$ & $\lnot P \land \lnot Q$ \\ \hline
      T & T & T & F & F & F & F \\
      T & F & T & F & F & T & F \\
      F & T & T & F & T & F & F \\
      F & F & F & T & T & T & T \\
    \end{tabular}
  \item
    The first distributive law: $P \land (Q \lor R) \equiv (P \land Q) \lor (P \land R)$.

    \begin{tabular}{c c c c c}
      $P$ & $Q$ & $R$ & $(Q \lor R)$ & $P \land (Q \lor R)$ \\ \hline
      T & T & T & T & T \\
      T & T & F & T & T \\
      T & F & T & T & T \\
      T & F & F & F & F \\
      F & T & T & T & F \\
      F & T & F & T & F \\
      F & F & T & T & F \\
      F & F & F & F & F \\
    \end{tabular}

    \begin{tabular}{c c c}
      $(P \land Q)$ & $(P \land R)$ & $(P \land Q) \lor (P \land R)$ \\ \hline
      T & T & T \\
      T & F & T \\
      F & T & T \\
      F & F & F \\
      F & F & F \\
      F & F & F \\
      F & F & F \\
      F & F & F \\
    \end{tabular}

    The second distributive law: $P \lor (Q \land R) \equiv (P \lor Q) \land (P \lor R)$.


    \begin{tabular}{c c c c c}
      $P$ & $Q$ & $R$ & $(Q \land R)$ & $P \lor (Q \land R)$ \\ \hline
      T & T & T & T & T \\
      T & T & F & F & T \\
      T & F & T & F & T \\
      T & F & F & F & T \\
      F & T & T & T & T \\
      F & T & F & F & F \\
      F & F & T & F & F \\
      F & F & F & F & F \\
    \end{tabular}

    \begin{tabular}{c c c}
      $P \lor Q$ & $P \lor R$ & $(P \lor Q) \land (P \lor Q)$ \\ \hline
      T & T & T \\
      T & T & T \\
      T & T & T \\
      T & T & T \\
      T & T & T \\
      T & F & F \\
      F & T & F \\
      F & F & F \\
      \end{tabular}
    \end{enumerate}
\item
  \begin{enumerate}
  \item

    \begin{equation*}
      \begin{aligned}
	\lnot (\lnot P \land \lnot Q) & \equiv \lnot \lnot P \lor \lnot \lnot Q && \text{DeMorgan's Law} \\
	& \equiv P \lor Q && \text{double negation law} \\
	\end{aligned}
    \end{equation*}

  \item
    \begin{equation*}
      \begin{aligned}
	(P \land Q) \lor (P \land \lnot Q) & \equiv P \land (Q \lor \lnot Q) && \text{Distributive law} \\
	& \equiv P \land (\text{tautology}) \\
	& \equiv P && \text{tautology laws} \\
	\end{aligned}
    \end{equation*}

  \item
    \begin{equation*}
      \begin{aligned}
	\lnot (P \land \lnot Q) \lor (\lnot P \land Q) & \equiv (\lnot P \lor Q) \lor (\lnot P \land Q) && \text{DeMorgan's law} \\
	& \equiv [(\lnot P \lor Q) \lor \lnot P] \land [(\lnot P \lor Q) \lor Q] && \text{Distributive law} \\
	& \equiv [(\lnot P \lor \lnot P) \lor Q] \land [\lnot P \lor (Q \lor Q)] && \text{Associative law and Commutative law} \\
	& \equiv (\lnot P \lor Q) \land (\lnot P \lor Q) && \text{Idempotent laws} \\
	& \equiv \lnot P \lor Q && \text{Idempotent laws} \\
	\end{aligned}
    \end{equation*}
  \end{enumerate}
\item
  TODO.
\item
  The Double negation law says that a negating a negation cancels it out, so
  \begin{equation*}
    \lnot P \land \lnot Q \equiv \lnot \lnot (\lnot P \land \lnot Q)
  \end{equation*}

  Using the first DeMorgan's law, we can negate the statement within the parenthesis:

  \begin{equation*}
    \begin{aligned}
      \lnot \lnot (\lnot P \land \lnot Q) & \equiv \lnot (\lnot \lnot P \lor \lnot \lnot Q) \\
      & \equiv \lnot (P \lor Q) \\
      \end{aligned}
  \end{equation*}

  Therefore, we've shown that $\lnot (P \lor Q) \equiv \lnot P \land \lnot Q$.

\item
  \begin{equation*}
    \begin{aligned}
      (P \land (Q \land R)) \land S & \equiv ((P \land Q) \land R) \land S \\
      & \equiv (P \land Q) \land (R \land S) \\
    \end{aligned}
  \end{equation*}

\item Each letter can have two values, true or false.  Each line in a truth table corresponds to a unique combination of the possible values for each letter.  Therefore, there are $2^n$ lines for $n$ letters.

\item From looking at the values, a good guess is that this involves $\lor$ because it's false only once.  Since the first line where both values are false is true, then that signals that one or both of them is negated, and the second line, where $P$ is false and $Q$ is true, and the formula is false suggests that it's $Q$ that's negated, so my guess is the formula is $P \lor \lnot Q$.  Here is the truth table:

  \begin{tabular}{c c c c}
    $P$ & $Q$ & $\lnot Q$ & $P \lor \lnot Q$ \\ \hline
    F & F & T & T \\
    F & T & F & F \\
    T & F & T & T \\
    T & T & F & T \\
    \end{tabular}

\item This is just the truth table for \textit{exclusive or} which we know from problem 3 equivalent to $(P \land \lnot Q) \lor (\lnot P \land Q)$.

\item To be valid, the premises cannot all be true without the conclusion being true as well.  So, if the conclusion is a tautology, then argument is valid, since whenever the premises are true (including never), we know that the conclusion is true as well.  If the conclusion is a contradiction, then the premises must also be a contradiction to be a valid argument.  If the premises are a tautology, then the conclusion must also be a tautology to be valid.  If the premises are a contradiction, then argument is always valid.
\end{enumerate}
\end{document}
